\section{CIFAR-10 Dataset}
CIFAR-10 is a widely used dataset in computer vision and machine learning. It contains 60,000 color images, each with a size of 32x32 pixels. The dataset is divided into 10 classes, such as airplane, car, bird, cat, deer, dog, frog, horse, ship, and truck. Each class has 6,000 images. In this project, only two classes are selected: \textbf{cat} (label 3) and \textbf{bird} (label 2), turning the problem into a binary classification task.

\subsection{Filtering the CIFAR-10 Dataset}
The class \texttt{FilteredCIFAR10Binary} takes a CIFAR-10 dataset and keeps only the samples with labels corresponding to cats and birds. 
Cats are relabeled as 0 and birds as 1. This simplifies the original 10-class CIFAR-10 dataset into a binary classification problem.
\begin{itemize}
    \item \texttt{\_\_init\_\_}: Stores the original dataset and selects only the samples with labels for cat (label 3) and bird (label 2).
    \item \texttt{\_\_getitem\_\_}: Retrieves an image and returns a binary label (0 for cat, 1 for bird).
\end{itemize}

\subsection{Creating Train and Test Loaders}
The class \texttt{CIFAR10CatBird} prepares the training and testing data loaders.

\begin{itemize}
    \item It uses \texttt{transforms} to resize images to 224x224, normalize them, and convert them to tensors.
    \item The \texttt{\_prepare\_loaders} method loads the original CIFAR-10 dataset and applies the binary filter.
    \item Two data loaders are created: one for training and one for testing.
\end{itemize}

\subsection{Adding a Poisoned Image}
The class \texttt{PoisonedDataset} allows the addition of one poisoned image into the training dataset.