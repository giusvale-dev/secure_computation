\documentclass{article}
\usepackage{graphicx}
\usepackage{amsmath}
\usepackage{hyperref}
\usepackage{listings}
\usepackage{xcolor}

\lstset{
  language=Python,
  basicstyle=\ttfamily\small,
  keywordstyle=\color{blue},
  commentstyle=\color{gray},
  stringstyle=\color{green!60!black},
  numbers=left,
  numberstyle=\tiny\color{gray},
  stepnumber=1,
  numbersep=5pt,
  showstringspaces=false,
  tabsize=4,
  breaklines=true,
  breakatwhitespace=true,
  frame=single
}

\author{Giuseppe Valente \\ Sapienza University of Rome \\ Secure Computation\\}
\date{2024/2025}
\title{Attack binary classifier using clean label poisoning feature collision on CIFAR-10 Dataset\\}
\begin{document}

\maketitle

\newpage

\tableofcontents

\newpage

\begin{abstract}
    This work presents an experiment where a poisoning attack is applied to a binary classifier trained on the CIFAR-10 dataset. 
    The model classifies images as either "Bird" or "Cat". Inspired by the Poison Frogs attack (Algorithm 1), the goal is to change a "Cat" image so that the model misclassifies it as "Bird".
\end{abstract}

\section{Introduction}
Machine learning models are vulnerable to different types of attacks. This work shows the \textsf{Targeted Clean-Label Poisoning Attacks}, which adds examples to the training set to manipulate the behavior of the model at test time.
This attack does not require control over the labeling of training data (\textbf{clean-label}). It is \textbf{targeted}, which means it can control the behavior of the classifier for a specific test instance without reducing the overall performance of the classifier.
This type of attack can be dangerous. For example, an attacker could add a poisoned image (properly labeled) to a training set for a face recognition engine, and control the identity of a chosen person at test time. 
Because the attacker does not need to control the labeling process, poisoned images can be added to the training set just by uploading them to the internet and waiting for them to be collected by a data scraping tool. 
The core goal of this work is to show the process \textsf{to poison} a "Cat" image in such a way that the model classifies it as a "Bird" while the poisoned image remains indistinguishable to the human eye.

\section{CIFAR-10 Dataset}
CIFAR-10 is a widely used dataset in computer vision and machine learning. It contains 60,000 color images, each with a size of 32x32 pixels. The dataset is divided into 10 classes, such as airplane, car, bird, cat, deer, dog, frog, horse, ship, and truck. Each class has 6,000 images. In this project, only two classes are selected: \textbf{cat} (label 3) and \textbf{bird} (label 2), turning the problem into a binary classification task.

\subsection{Filtering the CIFAR-10 Dataset}
The class \texttt{FilteredCIFAR10Binary} takes a CIFAR-10 dataset and keeps only the samples with labels corresponding to cats and birds. 
Cats are relabeled as 0 and birds as 1. This simplifies the original 10-class CIFAR-10 dataset into a binary classification problem.
\begin{itemize}
    \item \texttt{\_\_init\_\_}: Stores the original dataset and selects only the samples with labels for cat (label 3) and bird (label 2).
    \item \texttt{\_\_getitem\_\_}: Retrieves an image and returns a binary label (0 for cat, 1 for bird).
\end{itemize}

\subsection{Creating Train and Test Loaders}
The class \texttt{CIFAR10CatBird} prepares the training and testing data loaders.

\begin{itemize}
    \item It uses \texttt{transforms} to resize images to 224x224, normalize them, and convert them to tensors.
    \item The \texttt{\_prepare\_loaders} method loads the original CIFAR-10 dataset and applies the binary filter.
    \item Two data loaders are created: one for training and one for testing.
\end{itemize}

\subsection{Adding a Poisoned Image}
The class \texttt{PoisonedDataset} allows the addition of one poisoned image into the training dataset.
\section{Neural Network}

In this project, a fully connected neural network (FCN) is used for the binary classification task of distinguishing between birds and cats from the CIFAR-10 dataset.

While convolutional neural networks (CNNs) are typically more effective for image classification, the choice of an FCN is intentional and aligned with the experimental goals. The primary objective of this work is to study the effects of data poisoning attacks on neural networks, rather than to achieve state-of-the-art classification performance.

\subsection{Fully Connected Neural Network for Binary Classification}

The \texttt{Net} class defines a fully connected neural network using PyTorch for binary classification of CIFAR-10 images (bird vs. cat). The architecture and functionality are described as follows:

\begin{itemize}
    \item \textbf{Architecture:}
    \begin{itemize}
        \item Input images of size $3 \times 32 \times 32$ are flattened into vectors of size 3072.
        \item The model consists of three fully connected (\texttt{Linear}) layers:
        \begin{itemize}
            \item FC1: 3072 $\rightarrow$ 512 with ReLU activation.
            \item FC2: 512 $\rightarrow$ 256 with ReLU activation.
            \item FC3: 256 $\rightarrow$ 1 output (a single logit).
        \end{itemize}
    \end{itemize}

    \item \textbf{Forward Pass:}
    The \texttt{forward} method defines how input data flows through the network layers. After flattening, it is passed through the dense layers with ReLU activations, producing a single output used for binary classification.

    \item \textbf{Training (\texttt{train\_network} method):}
    \begin{itemize}
        \item The model is trained using a specified loss function and optimizer.
        \item For each epoch, the model performs a forward pass, computes the loss, backpropagates the gradients, and updates weights.
        \item After each epoch, the average loss is printed and recorded.
        \item The trained model is saved to the path defined by \texttt{TRAINED\_MODEL\_PATH}.
    \end{itemize}

    \item \textbf{Evaluation (\texttt{test} method):}
    \begin{itemize}
        \item The trained model is loaded and evaluated on the test dataset.
        \item Predictions are passed through a sigmoid function to obtain probabilities.
        \item Binary predictions are computed by thresholding at 0.5.
        \item The method calculates and prints the overall test accuracy.
    \end{itemize}
\end{itemize}

This architecture is suitable for simple binary classification tasks using flattened image vectors. While not as powerful as convolutional networks for image data, it provides a straightforward and interpretable baseline.



\section{Methodology}
The poisoning process follows the strategy described in the Poison Frogs paper. First, we train a binary classifier on clean CIFAR-10 data. Then, we choose one "Bird" image as the \textit{target}. We also select five "Cat" images to be poisoned.

The poisoning algorithm works by modifying each base image (Cat) so that the model’s internal output for the poisoned image becomes close to the output for the target image (Bird). This is done by minimizing the squared distance between their logits (model outputs before activation).

After poisoning, we retrain the model with the modified dataset and check if the poisoned images are now misclassified as "Bird".

\section{Experiment}
We use a neural network with three fully connected layers:
\begin{itemize}
    \item Input: 3072 features (flattened 3x32x32 image)
    \item FC1: 512 neurons with ReLU
    \item FC2: 256 neurons with ReLU
    \item Output: 1 neuron (logit for binary classification)
\end{itemize}

The CIFAR-10 dataset is split into 80\% training and 20\% testing. We apply a binary label where classes 2--7 are "Bird" (label 1), and the others are "Cat" (label 0). The model is trained using SGD with momentum, and binary cross-entropy loss.

After the clean training, we apply the Poison Frogs method with the following settings:
\begin{itemize}
    \item Learning rate: 0.01 for poison generation
    \item Number of poisoned samples: 5
    \item Iterations: 1000 per poisoned sample
\end{itemize}

\section{Results}
After clean training, the model achieved an accuracy of \textbf{86.37\%} on the test set. Five poisoned images were generated, and all five were misclassified as "Bird", even though their true label was "Cat". This shows the success of the poisoning attack.

Then we retrained the model with the poisoned dataset. After 10 more epochs, the model reached a slightly higher test accuracy of \textbf{86.59\%}, even though it now wrongly classified the poisoned images.

Figure~\ref{fig:poisoned} shows a comparison between the base and poisoned images. Figure~\ref{fig:misclassified} displays the poisoned images that were misclassified.

\begin{figure}[h]
    \centering
    \includegraphics[width=0.8\textwidth]{data/base_vs_poisoned.png}
    \caption{Base vs. Poisoned images. Left: original Cat. Right: modified (poisoned) version.}
    \label{fig:poisoned}
\end{figure}

\begin{figure}[h]
    \centering
    \includegraphics[width=0.6\textwidth]{data/missclassified.png}
    \caption{Poisoned images misclassified as "Bird".}
    \label{fig:misclassified}
\end{figure}

\section{Conclusion}
This experiment demonstrates that it is possible to manipulate a binary classifier by poisoning only a small number of training samples. The Poison Frogs method is effective at making the model learn wrong patterns that match the target image.

Although the overall accuracy remained high, the presence of poisoned samples showed that the model's decisions can be subtly changed, which is a serious problem for secure machine learning.

\end{document}
