\documentclass{article}
\usepackage{graphicx}
\usepackage{amsmath}
\usepackage{hyperref}
\usepackage{listings}
\usepackage{xcolor}

\lstset{
  language=Python,
  basicstyle=\ttfamily\small,
  keywordstyle=\color{blue},
  commentstyle=\color{gray},
  stringstyle=\color{green!60!black},
  numbers=left,
  numberstyle=\tiny\color{gray},
  stepnumber=1,
  numbersep=5pt,
  showstringspaces=false,
  tabsize=4,
  breaklines=true,
  breakatwhitespace=true,
  frame=single
}

\author{Giuseppe Valente \\ Sapienza University of Rome \\ Secure Computation\\}
\date{2024/2025}
\title{Attack binary classifier using clean label poisoning feature collision on CIFAR-10 Dataset\\}
\begin{document}

\maketitle

\newpage

\tableofcontents

\newpage

\begin{abstract}
    This work presents an experiment where a poisoning attack is applied to a binary classifier trained on the CIFAR-10 dataset. 
    The model classifies images as either "Bird" or "Cat". Inspired by the Poison Frogs attack (Algorithm 1), the goal is to slightly change a few "Cat" images so that the model misclassifies them as "Bird". 
    We will see the difference between the clean and the poisoned images, and we will measure the success of the model on the clean and the poisoned dataset.
\end{abstract}
\section{Introduction}
Machine learning models are vulnerable to different types of attacks, especially during training. One type of attack is called data poisoning, where an attacker changes part of the training data to influence the model’s behavior. These attacks happen at training time, they aim to manipulate
the performance of a system by inserting carefully constructed poison instances into the training data meaning they aim to control the behavior of a classifier on one specific test instance. For example, they manipulate a face recognition
engine to change the identity of one specific person, or manipulate a spam filter to allow/deny a specific email of the attacker’s choosing. The \textbf{clean label attacks} do not require control over the labeling function, the poisoned training data appear to be labeled correctly according to an expert observer. This makes the attacks not only difficult to detect, but opens the door for attackers to
succeed without any inside access to the data collection/labeling process. For example, an adversary could place poisoned images online and wait for them to be scraped by a bot that collects data from the web.

We focus on binary classification using the CIFAR-10 dataset. The model learns to separate "Bird" classes from "Cat" classes. The goal of this attack is to show how we can modify five "Cat" images in such a way that the model starts misclassifying them as "Bird".

\section{CIFAR-10 Dataset}
The CIFAR-10 dataset is a collection of images that is commonly used in machine learning and computer vision research. It contains 60,000 images in total, and these images are divided into 10 different categories. Each image is a color image with a size of 32x32 pixels. The categories in the dataset are:

\begin{enumerate}
    \setcounter{enumi}{-1}
    \item Airplane
    \item Automobile
    \item Bird
    \item Cat
    \item Deer
    \item Dog
    \item Frog
    \item Horse
    \item Ship
    \item Truck
\end{enumerate}

The dataset is split into two parts: a training set and a test set. The training set consists of 50,000 images, and the test set has 10,000 images.

\subsection{How the CIFAR-10 is loaded}
In this section are shown and discussed the techniques used to make the binary Dataloader.

\subsubsection{Transforms and Tensors in PyTorch}

Before using images in a neural network, they must be converted into a format suitable for computation. In PyTorch, this is done using \textit{transforms} and converting the data into \textit{tensors}.

\begin{itemize}
    \item \textbf{Transforms:}  
    A transform is a preprocessing step applied to data before it is used by a model. In this project, we use:
    \begin{verbatim}
    transforms.Compose([transforms.ToTensor()])
    \end{verbatim}
    This pipeline includes a single transformation:
    \begin{itemize}
        \item \texttt{ToTensor()}: Converts a PIL image (or NumPy array) into a PyTorch tensor.
    \end{itemize}
    Transforms can be expanded later to include data augmentation, normalization, resizing, etc.

    \item \textbf{Tensors:}  
    Tensors are the basic data structure in PyTorch. They are multi-dimensional arrays used to represent data.
    \begin{itemize}
        \item An RGB image is represented as a 3D tensor with shape \texttt{[Channels, Height, Width]}.
        \item For CIFAR-10, each image becomes a tensor of shape \texttt{[3, 32, 32]}.
    \end{itemize}
\end{itemize}

The transformation from image to tensor ensures that the data is compatible with PyTorch models.

\subsubsection{Batch Size and Number of Workers}

The \texttt{DataLoader} class in PyTorch handles how data is loaded in batches during training and testing. Two important parameters that affect performance are:

\begin{itemize}
    \item \textbf{Batch Size (\texttt{batch\_size}):} 
    This parameter defines how many samples are loaded and processed together in a single batch. A batch size of 64 is used in this project, meaning that each training step updates the model using 64 images at once. Larger batch sizes may speed up training but use more memory.

    \item \textbf{Number of Workers (\texttt{num\_workers}):} 
    This controls how many subprocesses are used to load the data in parallel. Using multiple workers (e.g., 2) can speed up training by reducing data loading time. A value of 0 loads data in the main thread, which is slower but simpler and sometimes more compatible.
\end{itemize}

These parameters help optimize the efficiency and performance of training by balancing memory usage and data loading speed.


\subsubsection{Custom CIFAR-10 Dataset Class for Binary Classification}

The Python class shown at listing \ref{lst:CustomCIFAR10Dataset} is designed to filter the CIFAR-10 dataset for a binary classification task involving only two categories: birds (label 2) and cats (label 3).
This class inherits from \texttt{torch.utils.data.Dataset}, enabling compatibility with PyTorch's \texttt{DataLoader}.
\begin{itemize}
    \item \textbf{Initialization (\texttt{\_\_init\_\_}):} The constructor accepts a CIFAR-10 dataset and an optional transformation. It loops each image-label pair, and only samples where the label is either 2 (bird) or 3 (cat). These are stored in internal lists.
    
    \item \textbf{Length (\texttt{\_\_len\_\_}):} This method returns the number of valid (bird or cat) images in the dataset.
    
    \item \textbf{Get item (\texttt{\_\_getitem\_\_}):} Given an index, this method retrieves the corresponding image and label. If a transformation function is provided, it is applied to the image before returning.
\end{itemize}
\begin{minipage}{\linewidth}
\begin{lstlisting}[caption=CustomCIFAR10Dataset, label={lst:CustomCIFAR10Dataset}]
class CustomCIFAR10Dataset(Dataset):
    def __init__(self, dataset, transform=None):
        self.dataset = dataset
        self.transform = transform
        # Filter images
        self.data = []
        self.targets = []
        for img, lbl in self.dataset:
            if lbl == 2 or lbl == 3:
                self.data.append(img)
                self.targets.append(lbl)

    def __len__(self):
        return len(self.data)

    def __getitem__(self, index):
        img, lbl = self.data[index], self.targets[index]

        if self.transform:
            img = self.transform(img)
        return img, lbl
\end{lstlisting}
\end{minipage}
\subsubsection{Binary CIFAR-10 Dataset Wrapper for Bird vs. Cat Classification}

The \texttt{BinaryCIFAR10Dataset} class is a utility for creating a binary classification dataset from CIFAR-10, specifically for distinguishing between birds and cats. It automates the loading, filtering, transformation, and labeling of the data, and creates PyTorch DataLoaders for easy integration into training workflows.

\begin{itemize}
    \item \textbf{Initialization (\texttt{\_\_init\_\_}):} 
    \begin{itemize}
        \item Downloads and loads the CIFAR-10 dataset (training and test sets).
        \item Applies a transformation that converts images to tensors.
        \item Filters the dataset using the defined \texttt{CustomCIFAR10Dataset} class, keeping only samples labeled as birds (2) or cats (3).
        \item Converts the class labels to binary:
              \begin{itemize}
                  \item Bird (label 2) becomes 1.
                  \item Cat (label 3) becomes 0.
              \end{itemize}
        \item Creates \texttt{DataLoader} instances for both training and testing with configurable batch size.
    \end{itemize}
    
    \item \textbf{Binary Label Conversion (\texttt{binary\_target}):}  
    This helper method maps the CIFAR-10 class index to a binary label. It returns 1 for birds and 0 for cats, enabling binary classification.
\end{itemize}
\begin{minipage}{\linewidth}
\begin{lstlisting}[caption=BinaryCIFAR10Dataset, label={lst:BinaryCIFAR10Dataset}]

class BinaryCIFAR10Dataset:

    def __init__(self, batch_size=64):

        # Apply transformations
        self.transform = transforms.Compose([transforms.ToTensor()])

        # Load CIFAR-10 dataset
        self.trainset = torchvision.datasets.CIFAR10(
            root='./data', train=True, download=True)
        
        self.testset = torchvision.datasets.CIFAR10(
            root='./data', train=False, download=True)

        # Create custom datasets with filtering and transformations
        self.trainset = CustomCIFAR10Dataset(self.trainset, transform=self.transform)

        self.testset = CustomCIFAR10Dataset(self.testset, transform=self.transform)

        # Convert to binary targets (1 = target_class, 0 = others)
        self.trainset.targets = [self.binary_target(t) for t in self.trainset.targets]
        
        self.testset.targets = [self.binary_target(t) for t in self.testset.targets]

        # Create DataLoaders
        self.trainloader = DataLoader(
            self.trainset, batch_size=batch_size, shuffle=True, num_workers=2)
        
        self.testloader = DataLoader(
            self.testset, batch_size=batch_size, shuffle=False, num_workers=2)
        
    def binary_target(self, target):
        
        return 1 if target == 2 else 0
\end{lstlisting}
\end{minipage}
\section{Neural Network}

In this project, a fully connected neural network (FCN) is used for the binary classification task of distinguishing between birds and cats from the CIFAR-10 dataset.

While convolutional neural networks (CNNs) are typically more effective for image classification, the choice of an FCN is intentional and aligned with the experimental goals. The primary objective of this work is to study the effects of data poisoning attacks on neural networks, rather than to achieve state-of-the-art classification performance.

\subsection{Fully Connected Neural Network for Binary Classification}

The \texttt{Net} class defines a fully connected neural network using PyTorch for binary classification of CIFAR-10 images (bird vs. cat). The architecture and functionality are described as follows:

\begin{itemize}
    \item \textbf{Architecture:}
    \begin{itemize}
        \item Input images of size $3 \times 32 \times 32$ are flattened into vectors of size 3072.
        \item The model consists of three fully connected (\texttt{Linear}) layers:
        \begin{itemize}
            \item FC1: 3072 $\rightarrow$ 512 with ReLU activation.
            \item FC2: 512 $\rightarrow$ 256 with ReLU activation.
            \item FC3: 256 $\rightarrow$ 1 output (a single logit).
        \end{itemize}
    \end{itemize}

    \item \textbf{Forward Pass:}
    The \texttt{forward} method defines how input data flows through the network layers. After flattening, it is passed through the dense layers with ReLU activations, producing a single output used for binary classification.

    \item \textbf{Training (\texttt{train\_network} method):}
    \begin{itemize}
        \item The model is trained using a specified loss function and optimizer.
        \item For each epoch, the model performs a forward pass, computes the loss, backpropagates the gradients, and updates weights.
        \item After each epoch, the average loss is printed and recorded.
        \item The trained model is saved to the path defined by \texttt{TRAINED\_MODEL\_PATH}.
    \end{itemize}

    \item \textbf{Evaluation (\texttt{test} method):}
    \begin{itemize}
        \item The trained model is loaded and evaluated on the test dataset.
        \item Predictions are passed through a sigmoid function to obtain probabilities.
        \item Binary predictions are computed by thresholding at 0.5.
        \item The method calculates and prints the overall test accuracy.
    \end{itemize}
\end{itemize}

This architecture is suitable for simple binary classification tasks using flattened image vectors. While not as powerful as convolutional networks for image data, it provides a straightforward and interpretable baseline.
\\We chose a neural network with three fully connected layers. This simple design is strong enough for our task and helps us focus on the main goal: studying how poisoning affects the learning process.




\begin{itemize}
    \item \textbf{More than one layer helps the model learn better:} One layer can only learn basic patterns. Using more layers helps the model understand more difficult patterns in the images.
    
    \item \textbf{Each layer has a different job:}
    \begin{itemize}
        \item The first layer finds simple things like color and edges.
        \item The second layer builds more useful shapes from these.
        \item The third layer makes the final decision: bird or cat.
    \end{itemize}
    
    \item \textbf{Simple is enough for our goal:} Because we want to study how poisoning changes training, we do not need a very advanced model.
    
    \item \textbf{Bigger networks are not needed here:} CIFAR-10 images are small, and we only have two classes. A bigger network would be harder to train and would not improve the results much.
\end{itemize}

In the listing \ref{lst:NeuralNetwork} is shown the class that implements the discussed functionalities:

\begin{minipage}{\linewidth}
    \begin{lstlisting}[caption=NeuralNetwork, label={lst:NeuralNetwork}]
        class Net(nn.Module):
        
            def __init__(self):
                
                super().__init__()
                self.flatten = nn.Flatten()
                self.fc1 = nn.Linear(3 * 32 * 32, 512)
                self.fc2 = nn.Linear(512, 256)
                self.fc3 = nn.Linear(256, 1)
                self.relu = nn.ReLU()
                
    \end{lstlisting}
\end{minipage}

\subsection{Understanding the Forward Pass of the Neural Network}

When an image is passed through the fully connected neural network (FCN), several operations are applied step by step. Each operation transforms the data until a final decision (output) is made. The following explains each part of the process in detail.

\begin{enumerate}
    \item \textbf{Flatten Layer:} 
    Images from CIFAR-10 are 3-dimensional (3 channels for color, 32×32 pixels). Before using them in a fully connected network, the image is converted to a 1D vector of length $3 \times 32 \times 32 = 3072$. This process is called \textit{flattening}.

    \item \textbf{First Linear Layer (FC1):}
    This layer takes the 3072 input values and maps them to 512 output values. This is done using a \textit{linear transformation}, which means each output is a weighted sum of the inputs, plus a bias term.

    \item \textbf{ReLU Activation:} 
    After the linear transformation, we apply a function called \textit{ReLU} (Rectified Linear Unit). This function replaces all negative values with zero. It helps the network learn non-linear patterns.

    \item \textbf{Second Linear Layer (FC2) + ReLU:} 
    The same process is repeated: the 512 values are transformed into 256, and ReLU is applied again.

    \item \textbf{Third Linear Layer (FC3):} 
    Finally, the 256 values are transformed into a single number. This number is called a \textit{logit}, which represents how strongly the model thinks the image belongs to the positive class (e.g., a bird).

    \item \textbf{Sigmoid Activation (at test time):} 
    In the testing phase, we apply the \textit{sigmoid} function to the output. This converts the logit into a probability between 0 and 1. If the probability is greater than 0.5, the image is classified as a bird; otherwise, it is classified as a cat.
\end{enumerate}

\begin{minipage}{\linewidth}
    \begin{lstlisting}[caption=Forward and Train, label={lst:ForwardTrain}]
        def forward(self, x):
        
                x = self.flatten(x)
                x = self.relu(self.fc1(x))
                x = self.relu(self.fc2(x))
                x = self.fc3(x)
                return x
        
        def train_network(self, trainloader, optimizer, criterion, num_epochs=100):
                
            self.train()
            train_losses = []
            for epoch in range(num_epochs):
                running_loss = 0.0
                self.train()
                for i, data in enumerate(trainloader, 0):
                    inputs, labels = data
                    labels = labels.float().unsqueeze(1)
                    optimizer.zero_grad()
                    outputs = self(inputs)
                    loss = criterion(outputs, labels)
                    loss.backward()
                    optimizer.step()
                    running_loss += loss.item()
                # Record average training loss
                avg_train_loss = running_loss / len(trainloader)
                train_losses.append(avg_train_loss)
                print(f"Epoch {epoch+1}: Train Loss = {avg_train_loss:.4f}")
            torch.save(self.state_dict(), TRAINED_MODEL_PATH)
            return train_losses
    \end{lstlisting}
\end{minipage}
                
\subsection{The \texttt{test} Method}

After training the neural network, we need to evaluate how well it performs on new data. The code in \ref{lst:test} shows the method responsible for this. It follows these steps:

\begin{itemize}
    \item \textbf{Loads the trained model:} It uses the weights saved during training to ensure the model is in its best state.
    
    \item \textbf{Switches to evaluation mode:} This disables certain training behaviors to make the results more stable.

    \item \textbf{Makes predictions on the test set:} The model runs over the test data and outputs values (logits).

    \item \textbf{Applies the sigmoid function:} This converts the logits to probabilities between 0 and 1.

    \item \textbf{Decides the predicted class:} If the probability is above 0.5, the model predicts class 1 (bird); otherwise, class 0 (cat).

    \item \textbf{Calculates accuracy:} The method compares predictions with the true labels and prints the percentage of correct predictions.
\end{itemize}

\begin{minipage}{\linewidth}
    \begin{lstlisting}[caption=Test, label={lst:test}]
            def test(self, testloader):
                self.eval()
                self.load_state_dict(torch.load(PATH))
                correct = 0
                total = 0
                with torch.no_grad():
                    for data in testloader:
                        images, labels = data
                        labels = labels.float().unsqueeze(1)  
        
                        outputs = self(images)
                        probs = torch.sigmoid(outputs)
                        predicted = (probs > 0.5).float()
        
                        total += labels.size(0)
                        correct += (predicted == labels).sum().item()
        
                accuracy = 100 * correct / total
                print(f'Accuracy of the network on the test images: {accuracy:.4f}%')
    \end{lstlisting}
\end{minipage}


\section{Methodology}
The poisoning process follows the strategy described in the Poison Frogs paper. First, we train a binary classifier on clean CIFAR-10 data. Then, we choose one "Bird" image as the \textit{target}. We also select five "Cat" images to be poisoned.

The poisoning algorithm works by modifying each base image (Cat) so that the model’s internal output for the poisoned image becomes close to the output for the target image (Bird). This is done by minimizing the squared distance between their logits (model outputs before activation).

After poisoning, we retrain the model with the modified dataset and check if the poisoned images are now misclassified as "Bird".

\section{Experiment}
We use a neural network with three fully connected layers:
\begin{itemize}
    \item Input: 3072 features (flattened 3x32x32 image)
    \item FC1: 512 neurons with ReLU
    \item FC2: 256 neurons with ReLU
    \item Output: 1 neuron (logit for binary classification)
\end{itemize}

The CIFAR-10 dataset is split into 80\% training and 20\% testing. We apply a binary label where classes 2--7 are "Bird" (label 1), and the others are "Cat" (label 0). The model is trained using SGD with momentum, and binary cross-entropy loss.

After the clean training, we apply the Poison Frogs method with the following settings:
\begin{itemize}
    \item Learning rate: 0.01 for poison generation
    \item Number of poisoned samples: 5
    \item Iterations: 1000 per poisoned sample
\end{itemize}

\section{Results}
After clean training, the model achieved an accuracy of \textbf{86.37\%} on the test set. Five poisoned images were generated, and all five were misclassified as "Bird", even though their true label was "Cat". This shows the success of the poisoning attack.

Then we retrained the model with the poisoned dataset. After 10 more epochs, the model reached a slightly higher test accuracy of \textbf{86.59\%}, even though it now wrongly classified the poisoned images.

Figure~\ref{fig:poisoned} shows a comparison between the base and poisoned images. Figure~\ref{fig:misclassified} displays the poisoned images that were misclassified.

\begin{figure}[h]
    \centering
    \includegraphics[width=0.8\textwidth]{data/base_vs_poisoned.png}
    \caption{Base vs. Poisoned images. Left: original Cat. Right: modified (poisoned) version.}
    \label{fig:poisoned}
\end{figure}

\begin{figure}[h]
    \centering
    \includegraphics[width=0.6\textwidth]{data/missclassified.png}
    \caption{Poisoned images misclassified as "Bird".}
    \label{fig:misclassified}
\end{figure}

\section{Conclusion}
This experiment demonstrates that it is possible to manipulate a binary classifier by poisoning only a small number of training samples. The Poison Frogs method is effective at making the model learn wrong patterns that match the target image.

Although the overall accuracy remained high, the presence of poisoned samples showed that the model's decisions can be subtly changed, which is a serious problem for secure machine learning.

\end{document}
