\section{MobileNetV2}
MobileNetV2 is a lightweight convolutional neural network that is efficient and fast, especially useful for tasks where computational resources are limited. 
It was originally trained on the ImageNet dataset to classify images into 1,000 categories. In this project, we reuse this pretrained model to classify only two categories.
The goal of this code is to create a neural network that can classify images into two categories: \textit{Cat} or \textit{Bird}. 
It uses a pretrained version of the MobileNetV2 model and modifies it for binary classification.

\subsection{Key Implementation Steps}
\begin{itemize}
    \item \textbf{Loading Pretrained Weights:} The model uses the default pretrained weights from \texttt{MobileNet\_V2\_Weights}, which come from training on ImageNet.
    
    \item \textbf{Freezing Feature Extractor:} The parameter \texttt{freeze\_features} decides whether to freeze the feature extractor layers. When frozen, these layers do not update during training, allowing faster training and avoiding overfitting on small datasets.

    \item \textbf{Replacing the Classifier:} The original classifier in MobileNetV2 outputs 1,000 classes. Here, it is replaced with a new linear layer that outputs 2 classes (Cat and Bird).

    \item \textbf{Forward Method:} The \texttt{forward} function simply passes the input image through the base model, which includes both the feature extractor and the new classifier.
\end{itemize}


